% Template KLTN cho SV trường ĐHKHTN
% Liên hệ: nqminh@fit.hcmus.edu.vn
% Last update: 08/06/2016

% Chú ý: đọc các phần chú ý đóng khung của file này và chỉnh lại cho phù hợp.
% Trước khi build, xóa hết các file được tạo ra trong quá trình build trước đó, và build theo thứ tự: BIB > PDF > PDF.
% Nếu cập nhật tài liệu tham khảo, cũng cần build lại theo cách trên.

\documentclass[oneside,a4paper,14pt]{extreport}

% Font tiếng Việt
\usepackage[T5]{fontenc}
\usepackage[utf8]{inputenc}
\DeclareTextSymbolDefault{\DH}{T1}

% Tài liệu tham khảo
\usepackage[
	sorting=nty,
	backend=bibtex,
	defernumbers=true]{biblatex}
\usepackage[unicode]{hyperref} % Bookmark tiếng Việt
\addbibresource{References/references.bib}

\makeatletter
\def\blx@maxline{77}
\makeatother

% Chèn hình, các hình trong luận văn được để trong thư mục Images/
\usepackage{graphicx}
\graphicspath{ {Images/} }

% Chèn và định dạng mã nguồn
\usepackage{listings}
\usepackage{color}
\definecolor{codegreen}{rgb}{0,0.6,0}
\definecolor{codegray}{rgb}{0.5,0.5,0.5}
\definecolor{codepurple}{rgb}{0.58,0,0.82}
\definecolor{backcolour}{rgb}{0.95,0.95,0.92}
\lstdefinestyle{mystyle}{
    backgroundcolor=\color{backcolour},
    commentstyle=\color{codegreen},
    keywordstyle=\color{magenta},
    numberstyle=\tiny\color{codegray},
    stringstyle=\color{codepurple},
    basicstyle=\footnotesize,
    breakatwhitespace=false,
    breaklines=true,
    captionpos=b,
    keepspaces=true,
    numbers=left,
    numbersep=5pt,
    showspaces=false,
    showstringspaces=false,
    showtabs=false,
    tabsize=2
}
\lstset{style=mystyle}

% Chèn và định dạng mã giả
\usepackage{amsmath}
\usepackage{algorithm}
\usepackage[noend]{algpseudocode}
\makeatletter
\def\BState{\State\hskip-\ALG@thistlm}
\makeatother

% Bảng biểu
\usepackage{multirow}
\usepackage{array}
\newcolumntype{L}[1]{>{\raggedright\let\newline\\\arraybackslash\hspace{0pt}}m{#1}}
\newcolumntype{C}[1]{>{\centering\let\newline\\\arraybackslash\hspace{0pt}}m{#1}}
\newcolumntype{R}[1]{>{\raggedleft\let\newline\\\arraybackslash\hspace{0pt}}m{#1}}

% Đổi tên mặc định
\renewcommand{\chaptername}{Chương}
\renewcommand{\figurename}{Hình}
\renewcommand{\tablename}{Bảng}
\renewcommand{\contentsname}{Mục lục}
\renewcommand{\listfigurename}{Danh sách hình}
\renewcommand{\listtablename}{Danh sách bảng}
\renewcommand{\appendixname}{Phụ lục}

% Dãn dòng 1.5
\usepackage{setspace}
\onehalfspacing

% Thụt vào đầu dòng
\usepackage{indentfirst}

% Canh lề
\usepackage[
  top=30mm,
  bottom=25mm,
  left=30mm,
  right=20mm,
  includefoot]{geometry}

% Trang bìa
\usepackage{tikz}
\usetikzlibrary{calc}
\newcommand\HRule{\rule{\textwidth}{1pt}}

% ========================================================================================= %
% CHÚ Ý: Thông tin chung về KLTN - sinh viên điền vào đây để tự động update các trang khác  %
% ========================================================================================= %
\newcommand{\tenSV}{Nguyễn~Văn~A~-~Trần~Văn~B} % Dấu ~ là khoảng trắng không được tách (các chữ nối với nhau bằng dấu ~ sẽ nằm cùng 1 dòng
\newcommand{\mssv}{1234567}
\newcommand{\tenKL}{Sử~dụng~LaTeX trong Khoá~luận~tốt~nghiệp} % Chú ý dấu ~ trong tên khóa luận
\newcommand{\tenGVHD}{Tên~Giáo~Viên}
\newcommand{\tenBM}{Công nghệ tri thức}

\begin{document}

\begin{titlepage}

	\begin{center}
		%ĐẠI HỌC QUỐC GIA THÀNH PHỐ HỒ CHÍ MINH\\
		TRƯỜNG ĐẠI HỌC KHOA HỌC TỰ NHIÊN\\ \textbf{KHOA CÔNG NGHỆ THÔNG TIN}\\[2cm]

		{ \Large \bfseries Bùi Huy Thông\\[2cm] }

		%Tên đề tài Khóa luận tốt nghiệp/Đồ án tốt nghiệp

		{ \Large \bfseries QUẢN LÝ VÀ CHIA SẺ NỘI DUNG SỐ \\ PHI TẬP TRUNG \\[3cm]}

		%Chọn trong các dòng sau
		\large LUẬN VĂN THẠC SĨ KHOA HỌC MÁY TÍNH\\
		%\large ĐỒ ÁN TỐT NGHIỆP CỬ NHÂN\\
		%\large THỰC TẬP TỐT NGHIỆP CỬ NHÂN\\
		%Đưa vào dòng này nếu thuộc chương trình Chất lượng cao, hoặc lớp Cử nhân tài năng
		%\large CHƯƠNG TRÌNH CHÍNH QUY\\
		%\large CHƯƠNG TRÌNH CHẤT LƯỢNG CAO\\
		%\large CHƯƠNG TRÌNH CỬ NHÂN TÀI NĂNG\\[2cm]

		\begin{tikzpicture}[remember picture, overlay]
			\draw[line width = 2pt] ($(current page.north west) + (2cm,-2cm)$) rectangle ($(current page.south east) + (-1.5cm,2cm)$);
		\end{tikzpicture}

		\vfill
		Tp. Hồ Chí Minh, tháng MM/YYYY

	\end{center}

	\pagebreak

	\begin{center}

		TRƯỜNG ĐẠI HỌC KHOA HỌC TỰ NHIÊN\\ \textbf{KHOA CÔNG NGHỆ THÔNG TIN}\\[2cm]

		{\large \bfseries Bùi Huy Thông - 2611030\\}

		%Tên đề tài Khóa luận tốt nghiệp/Đồ án tốt nghiệp

		{ \Large \bfseries QUẢN LÝ VÀ CHIA SẺ NỘI DUNG SỐ \\ PHI TẬP TRUNG \\[2cm]}

		%Chọn trong các dòng sau
		\large LUẬN VĂN THẠC SĨ KHOA HỌC MÁY TÍNH\\
		%\large ĐỒ ÁN TỐT NGHIỆP CỬ NHÂN\\
		%Đưa vào dòng này nếu thuộc chương trình Chất lượng cao, hoặc lớp Cử nhân tài năng
		%\large CHƯƠNG TRÌNH CHÍNH QUY\\[2cm]
		%\large CHƯƠNG TRÌNH CHẤT LƯỢNG CAO\\[2cm]
		%\large CHƯƠNG TRÌNH CỬ NHÂN TÀI NĂNG\\[2cm]

		\textbf{NGƯỜI HƯỚNG DẪN}\\
		PGS.TS. Nguyễn Đình Thúc\\

		\begin{tikzpicture}[remember picture, overlay]
			\draw[line width = 2pt] ($(current page.north west) + (2cm,-2cm)$) rectangle ($(current page.south east) + (-1.5cm,2cm)$);
		\end{tikzpicture}

		\vfill
		Tp. Hồ Chí Minh, tháng MM/YYYY

	\end{center}

\end{titlepage}
% Sasu trang Title, các bạn chèn nhận xét gủa GVHD và GVPB. Nhận xét sẽ được giáo vụ phát sau buổi bảo vệ để các bạn đóng quyển.

\pagenumbering{roman} % Đánh số i, ii, iii, ...

%\addcontentsline{toc}{chapter}{Lời cam đoan}
%\chapter*{Lời cam đoan}
\label{reassurances}

Tôi xin cam đoan đây là công trình nghiên cứu của riêng tôi. Các số liệu và kết
quả nghiên cứu trong luận văn này là trung thực và không trùng lặp với các đề
tài khác.

\addcontentsline{toc}{chapter}{Lời cảm ơn}
\chapter*{Lời cảm ơn}
\label{thanks}

Tôi xin chân thành cảm ơn ...

\addcontentsline{toc}{chapter}{Đề cương chi tiết}
\include{Appendix/decuong}

% Mục lục, danh sách hình, danh sách bảng
\addcontentsline{toc}{chapter}{Mục lục}
\tableofcontents
\listoffigures
\listoftables

\addcontentsline{toc}{chapter}{Tóm tắt}
\include{Appendix/tomtat}

\clearpage

\pagenumbering{arabic} % Đánh số 1, 2, 3, ...

% Các chương nội dung
\chapter{Giới thiệu}
\label{Chapter1}

%Tóm tắt luận văn được trình bày nhiều nhất trong 24 trang in trên hai mặt giấy, cỡ chữ Times New Roman 11 của hệ soạn thảo Winword hoặc phần mềm soạn thảo Latex đối với các chuyên ngành thuộc ngành Toán.

%Mật độ chữ bình thường, không được nén hoặc kéo dãn khoảng cách giữa các chữ.
%Chế độ dãn dòng là Exactly 17pt.
%Lề trên, lề dưới, lề trái, lề phải đều là 1.5 cm.
%Các bảng biểu trình bày theo chiều ngang khổ giấy thì đầu bảng là lề trái của trang.
%Tóm tắt luận án phải phản ảnh trung thực kết cấu, bố cục và nội dung của luận án, phải ghi đầy đủ toàn văn kết luận của luận án.
%Mẫu trình bày trang bìa của tóm tắt luận văn (phụ lục 1).

\chapter{Các công trình liên quan}
\label{Chapter2}
\subsection{Blockchain}
Trong những năm gần đây, với sự mở rộng và phát triển nhanh chóng của các loại
tiền ảo (cryptocurrency) như Bitcoin[], Etherum[], Zcash[], ...etc, khiến cho
mức độ quan tâm đến công nghệ nền tảng của chúng, blockchain, tăng lên đáng kể.
Trên thực tế, blockchain đã cho thấy công nghệ này không chỉ đóng vai trò quan
trọng trong các lĩnh vực liên quan đến tài chính, mà các ứng dụng của nó đã và
đang dần phổ biến rộng rãi trên nhiều lĩnh vực phi tài chính khác. Shen[] đã
tổng hợp các ứng dụng của blockchain trong việc phát triển đô thị thông minh và
sắp xếp chúng thành 9 danh mục, bao gồm: quản lý nhà nước(governance and
citizen engagement), giáo dục, chăm sóc sức khỏe, kinh tế, giao thông vận tải,
năng lượng, quản lý nước và chất thải, công trình công cộng và bảo vệ môi
trường. Jaroodi[] cũng đã chỉ ra các lợi ích cũng như thách thức của việc sử
dụng blockchain trong các lĩnh vực: tài chính, chăm sóc sức khỏe, các hoạt động
thương mại, sản xuất, năng lượng, nông nghiệp và thực phẩm, tự động hóa, xây
dựng, truyền thông và các ứng dụng giải trí.
\subsubsection{Blockchain trong việc bảo vệ dữ liệu và thông tin cá nhân}
Bảo mật thông tin cá nhân khi sử dụng internet đã và đang là một vấn đề nan
giải khi các mạng xã hội liên tục thu thập thông tin người dùng, bao gồm các
thông tin cá nhân, hoạt động và thói quen. Người sử dụng các mạng xã hội gặp
khó khăn trọng việc quản lý các loại thông tin mà nhà cung cấp dịch vụ được
quyền thu thập và mục đích sử dụng các thông tin này và thường không thể rút
lại các quyền truy cập đã cho phép. Zyskind[] đề xuất một ý tưởng lưu trữ các
chính sách truy cập thông tin trên một Blockchain và để cho các node của
Blockchain truy cập thông tin từ một bảng băm phân tán (Distributed hash
table). Khi người dùng muốn cấp hoặc hủy bỏ quyền truy cập vào thông tin cá
nhân, Blockchain đóng vai trò như một nhà phân phối chỉ cho phép bên thứ 3 truy
cập các thông tin đã được cấp quyền. Wang[] đề xuất một mô hình kết hợp hệ
thống lưu trữ phân tán (decentrailize storage system), Etherum Blockchain và kĩ
thuật mã hóa dựa trên thuộc tính (attribute-based ecryption). Mô hình này cho
phép người dùng tìm kiếm trên các thông tin đã mã hóa dựa vào các hợp đồng
thông minh (smart contract) của Etherum. Đồng thời cũng cho phép người sở hữu
thông tin phân phối các khóa bí mật cho người dùng và chia sẻ thông tin thông
qua các chính sách truy cập đặc tả. \\
\subsubsection{Blockchain trong các hệ thống IOT}

\subsubsection{Các hệ thống lưu trữ phân tán - (Decentralize storage systems)}
Các hệ thống lưu trữ phân tán hướng đến việc chia nhỏ dữ liệu và phân phối
chúng lên các node trong một mạng lưới có sẵn thay vì dồn nén tất cả dữ liệu
trong một server duy nhất. Dựa vào bảng A, ta nhận thấy các hệ thống lưu trữ
phân tán có một số ưu điểm so với cách lưu trữ tập trung truyền thống.
\begin{itemize}
	\item Khi một hệ thống bị tấn công, việc phân tán dữ liệu sẽ giảm thiểu được rủi ro
	      về thông tin bị lộ một cách toàn vẹn.
	\item Ngoài ra, tính tồn tại của một hệ thống lưu trữ phân tán cũng cao hơn, khi một
	      hoặc nhiều node bị ngắt kết nối, các node còn lại có thể được lập trình để hoạt
	      động độc lập. Điều này tốt hơn nhiều so với việc một hệ thống lưu trữ tập trung
	      sẽ ngưng hoạt động khi server lưu trữ bị ngắt kết nối.
	\item Cuối cùng, một hệ thống lưu trữ phi tập trung sẽ thân thiện hơn với người dùng
	      thông qua việc cho phép người sử dụng dịch vụ trả phí theo dung lượng đã sử
	      dụng thay vì phải trả trước để mua dung lượng như cách làm hiện tại của các hệ
	      thống lưu trữ tập trung.
\end{itemize}

\subsection{Mã hóa bất đối xứng}

\subsection{Chữ kí điện tử}

\chapter{Phương pháp}
\subsection{Tạo và chuyển dữ liệu cho chủ dữ liệu}

\subsection{Chia sẻ dữ liệu ngang hàng}

\chapter{Ứng dụng và cài đặt}
\subsection{Web Application (cho nhà cung cấp nội dung)}

\subsection{Mobile Application (cho người dùng)}

% Công trình của tác giả (nếu không có thì comment 02 dòng dưới)
\addcontentsline{toc}{chapter}{Danh mục công trình của tác giả}
\chapter*{Danh mục công trình của tác giả}
\label{Appendix1}

\begin{enumerate}
	\item Tạp chí ABC
	\item Tạp chí XYZ
\end{enumerate}

% In tài liệu tham khảo
\addcontentsline{toc}{chapter}{Tài liệu tham khảo}
\printbibheading[title={Tài liệu tham khảo}]

\printbibliography[heading=subbibliography, title={Tiếng Việt}, keyword=Viet, resetnumbers=true]

\DeclareNameAlias{sortname}{last-first}
\DeclareNameAlias{default}{last-first}

\printbibliography[heading=subbibliography, title={Tiếng Anh}, notkeyword=Viet, resetnumbers=4]
% ===================================================================== %
% CHÚ Ý: phải gán lại resetnumbers=số tài liệu tham khảo tiếng Việt + 1 %
% ===================================================================== %

% Phần phụ lục
%\appendix

\chapter{Ngữ pháp tiếng Việt}
\label{Appendix1}

Đây là phụ lục.
%\chapter{Ngữ pháp tiếng Nôm}
\label{Appendix2}

Đây là phụ lục 2.

\end{document}
