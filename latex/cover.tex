\begin{titlepage}
    \centering
    \vspace*{1cm}
    
    % University/School name
    {\Large\textbf{ĐẠI HỌC KHOA HỌC TỰ NHIÊN TPHCM}\par}
    {\Large\textbf{KHOA CÔNG NGHỆ THÔNG TIN}\par}
    \vspace{1.5cm}
    
    % Logo (optional)
    % \includegraphics[width=0.3\textwidth]{logo.png}\\
    \vspace{1.5cm}
    
    % Title
    {\Huge\textbf{PHƯƠNG PHÁP TOÁN TRONG TIN HỌC VÀ GIẢI THUẬT}\par}
    \vspace{1cm}
    {\LARGE\textbf{ĐỒ ÁN 3: LỖ HỎNG NGUYÊN TRONG PHÉP TOÁN $x \cdot n$}\par}
    \vspace{2cm}
    
    % Course information
    {\Large Nhóm 5\par}
    \vspace{0.5cm}
    
    % Team members
    \begin{center}
    \vspace{4pt} 
    \renewcommand{\arraystretch}{1.5} 
    \begin{tabular}{|>{\centering\arraybackslash}m{8cm}|>{\centering\arraybackslash}m{4cm}|}
        \hline
        \textbf{Họ và tên} & \textbf{MSSV} \\
        \hline
        Phạm Thị Anh Đào & 23C11003 \\
        \hline
        Lê Minh Trí & 24C11030 \\
        \hline
        Dương Tiến Vinh & 24C11034 \\
        \hline
        Bùi Thế Vinh & 24C11070 \\
        \hline
        Đỗ Hoài Nam & 24C12021 \\
        \hline
        Trần Quang Duy & 24C12027 \\
        \hline    
    \end{tabular}
\end{center}

    \vspace{3cm}
    
    {\Large Giảng viên hướng dẫn: PGS.TS. Trần Đan Thư\par}
    \vspace{1cm}
    % Date
    {\Large TP. Hồ Chí Minh, Tháng 03 Năm 2025\par}
    
\end{titlepage}

% Main content pages begin here
\LARGE\textbf{Nội dung đồ án 3: }
Tập trung vào phép toán $x \cdot n$ với $x$ là số double và $n \in \mathbb{N}^{+}$ (nguyên dương) có khả năng lấy giá trị lớn

\renewcommand{\labelenumi}{\alph{enumi})}
\begin{enumerate}
    \item Tìm 3 trường hợp $(x,n)$ có thể xác định rõ giá trị $x \cdot n$ bằng cách tính trên giấy ra kết quả chính xác, trong khi đó tính bằng Excel hay hàm nhân có sẵn ra kết quả sai lệch đáng kể ấn tượng
    
    \item Cài đặt thuật toán lũy thừa nhanh và chạy thử ra kết quả đúng cho 3 ví dụ trên
    
    \item Minh họa cho thuật toán tuyến tính $O(n)$ tính $x \cdot n$ rất chậm
    
\end{enumerate}