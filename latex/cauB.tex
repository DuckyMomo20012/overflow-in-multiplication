\section{Cài đặt thuật toán lũy thừa nhanh và chạy thử ra kết quả
đúng cho 3 ví dụ trên)}
\subsection{Thuật toán nhân truyền thống}
\begin{verbatim}
def linear_multiply(x, n):
    """
    Multiply a number by n using repeated addition.
    Time complexity: O(n)
    """
    result = 0
    for _ in range(n):
        result += x
    return result
\end{verbatim}

Phép nhân $x \cdot n$ được hiểu là việc cộng số $x$ với chính nó $n$ lần. Ví dụ, $5 \cdot 3$ được tính bằng cách lấy $5 + 5 + 5 = 15$.

\subsubsection{}{Các bước thực hiện}
Hàm \texttt{linear\_multiply} triển khai nguyên lý này một cách trực tiếp:
\begin{enumerate}
    \item Khởi tạo biến \texttt{result = 0} để lưu trữ kết quả phép nhân
    \item Sử dụng vòng lặp for trong Python để lặp đúng $n$ lần
    \item Trong mỗi vòng lặp, cộng giá trị $a$ vào biến result: \texttt{result += x}
    \item Sau khi hoàn thành $n$ vòng lặp, kết quả cuối cùng chính là $x \cdot n$
\end{enumerate}
\subsubsection{Độ phức tạp thuật toán}
Thuật toán này có độ phức tạp $O(n)$ vì:
\begin{itemize}
   \item Số lần lặp tỷ lệ thuận với giá trị của $n$
   \item Khi $n$ tăng lên gấp đôi, thời gian thực hiện cũng tăng gấp đôi
   \item Mỗi vòng lặp chỉ thực hiện một phép cộng đơn giản $O(1)$
\end{itemize}
Điều này làm cho thuật toán trở nên kém hiệu quả với các giá trị $n$ lớn, nhưng minh họa rõ ràng mối liên hệ giữa phép nhân và phép cộng lặp lại trong toán học cơ bản.

\subsection{Thuật toán nhân nhanh}
\begin{verbatim}
def fast_multiply(x, binary_n):
    """
    Multiply x number by n using binary multiplication 
    technique.
    Time complexity: O(log(n))
    
    Parameters:
    x: number to multiply
    binary_n: string of '0's and '1's representing n in
    binary
    """
    result = 0
    
    # Get index for processing from right to left
    idx = len(binary_n) - 1
    
    # Process binary digits from right to left 
    while idx >= 0:
        bit = binary_n[idx]
        if bit == '1':
            result += x
        x += x  # Double x for next bit position
        idx -= 1
    return result
\end{verbatim}
\subsubsection{Nguyên lý hoạt động của thuật toán}

Thuật toán này tận dụng biểu diễn nhị phân của số nguyên để thực hiện phép nhân hiệu quả. Thay vì cộng số $x$ với chính nó $n$ lần như phương pháp nhân O(n), phương pháp này xử lý từng bit của số $n$  và thực hiện các bước tương ứng.

\subsubsection{Các bước thực hiện}

Khi thực hiện \texttt{fast\_multiply}, thuật toán hoạt động như sau:

\begin{enumerate}[label=\alph*)]
    \item Khởi tạo biến \texttt{result = 0} để lưu trữ kết quả phép nhân và chuyển $n$ sang dạng binary: \texttt{binary\_n} 
    \item Trong khi \texttt{len(binary\_n) >= 0}, thuật toán:
        \begin{enumerate}[label=b\arabic*)]
            \item Kiểm tra bit cuối cùng của $n$
            \item Nếu bit là 1, cộng giá trị hiện tại của $x$ vào $result$
            \item Nếu bit là 0, không cộng giá trị hiện tại của $x$ vào $result$
            \item Nhân đôi giá trị của $x$ (tương đương với phép dịch trái một bit)
        \end{enumerate}
\end{enumerate}

\noindent Về bản chất, thuật toán này cộng $x \cdot 2^i$ vào kết quả cho mỗi vị trí bit thứ $i$ mà bit trong $n$ là 1.

\noindent Giải thích: Khi một số nguyên $n$ được biểu diễn trong hệ nhị phân, mỗi bit đại diện cho một lũy thừa của 2. Ví dụ, số $n=13$ được biểu diễn là $1101_2$, có nghĩa là:
\begin{align*}
13 = 1 \cdot 2^3 + 1 \cdot 2^2 + 0 \cdot 2^1 + 1 \cdot 2^0 
\end{align*}
Khi nhân $x$ với $n$, chúng ta có thể viết: 
\begin{align*}
x \cdot n &= x \cdot 13 = x \cdot (1 \cdot 2^3 + 1 \cdot 2^2 + 0 \times 2^1 + 1 \cdot 2^0) \\
&= x \cdot 1 \cdot 2^3 + x \cdot 1 \cdot 2^2 + x \cdot 0 \cdot 2^1 + x \cdot 1 \cdot 2^0
\end{align*}
\noindent Do đó, chỉ có những bit là 1 mới cần cộng vào kết quả
\section{Độ phức tạp thuật toán}
Phương pháp này có độ phức tạp thời gian là \( O(\log n) \) vì:
\begin{enumerate}
    \item Vòng lặp chính chạy theo số bit trong biểu diễn nhị phân của \( n \).
    \item Một số nguyên \( n \) có khoảng \( \log_2(n) \) bit trong biểu diễn nhị phân của nó. \\
    Chứng minh: 

    Nếu có $m$ bit, sẽ biểu diễn được các số từ $0$ đến $2^m - 1$ trong hệ nhị phân (do mỗi bit có 2 trạng thái là 0 và 1, do đó m bit sẽ là $2^m$ trạng thái). Do đó, nếu biểu diễn một số $n$ trong hệ nhị phân cần một số lượng bit $m$ sao cho:
    
    \[
    2^m - 1 \geq n
    \]
    
    hay tương đương với:
    
    \[
    m \geq \log_2 (n + 1).
    \]
    Khi $n$ đủ lớn, số lượng bit m cần để biểu diễn n là   \[m \geq \log_2 (n + 1) \approx \log_2 (n) \] \\
    Vậy một số nguyên \( n \) có khoảng \( \log_2(n) \) bit trong biểu diễn nhị phân của nó.

    \item Do có khoảng \( \log_2(n) \) bit trong biểu diễn nhị phân của \( n \), thuật toán sẽ lặp \( \log_2(n) \) lần.
    \end{enumerate}